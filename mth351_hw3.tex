\documentclass[12pt,letterpaper]{article}
\usepackage{amsmath} % just math
\usepackage{amssymb} % allow blackboard bold (aka N,R,Q sets)
\usepackage{ulem}
\usepackage{graphicx}
\usepackage{float}
\linespread{1.6}  % double spaces lines
\usepackage[left=1in,top=1in,right=1in,bottom=1in,nohead]{geometry}
\setcounter{section}{2}
\begin{document}
\begin{flushright}
\end{flushright}
\begin{flushleft}
\textbf{Eric Zounes} \\
\today \\ 
Sec 3.1: 10*, 15ab \\
Sec 3.2: 3, 9*, 13 \\
Sec 3.3: 8* \\
Sec 3.4: 7, 8, 9*, 13, 14 \\
Sec 3.5: 7*, 8 \\
\end{flushleft}
\subsection{The Bisection Method} 
	\begin{enumerate} 
		\item[10.] Consider the equation $e^{-x} = sin(x)$. Find an interval $[a,b]$ that contains the smallest positive root. Estimate the number of midpoints c needed to obtain an approximate root that is accurate within an error tolerance of $10^{-1)}$. \\
			
\end{document}
